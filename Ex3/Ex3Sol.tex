\title{Statistical learning and data analysis\\\large Assignment 3 - Norms, Regression and Regularization}
\author{NathanP}
\date{\today}
\documentclass{article}
\usepackage{amsmath, amssymb}
\begin{document}
\maketitle
\section*{1. Norms}

\subsection*{(a) Show that equation (1) defines a norm for \( p \geq 1 \)}
To be a norm, the function \( \|\cdot\|_p \) must satisfy:
\begin{enumerate}
    \item \textbf{Positive definiteness:} \( \|x\|_p \geq 0 \) and \( \|x\|_p = 0 \iff x = 0 \)
    \item \textbf{Homogeneity:} \( \|\alpha x\|_p = |\alpha|\|x\|_p \)
    \item \textbf{Triangle inequality:} \( \|x + y\|_p \leq \|x\|_p + \|y\|_p \)
\end{enumerate}

\textbf{1. Positive definiteness:} All \( |x_i|^p \geq 0 \), so the sum is non-negative. If \( \|x\|_p = 0 \), then \( |x_i|^p = 0 \Rightarrow x_i = 0 \ \forall i \), so \( x = 0 \).

\textbf{2. Homogeneity:}
\[
\|\alpha x\|_p = \left( \sum_{i=1}^n |\alpha x_i|^p \right)^{1/p} = \left( |\alpha|^p \sum_{i=1}^n |x_i|^p \right)^{1/p} = |\alpha| \cdot \|x\|_p
\]

\textbf{3. Triangle inequality:} This follows from \textbf{Minkowski's inequality}:
\[
\|x + y\|_p \leq \|x\|_p + \|y\|_p \quad \text{for } p \geq 1
\]

Hence, \( \|\cdot\|_p \) is a norm for \( p \geq 1 \).

\subsection*{(b) Show that equation (1) does not define a norm for \( p < 1 \)}

For \( 0 < p < 1 \), the triangle inequality does \textbf{not} hold.

Let \( x = (1, 0) \), \( y = (0, 1) \), \( p = 0.5 \):
\[
\|x + y\|_p = \|(1, 1)\|_p = (1^{0.5} + 1^{0.5})^{1/0.5} = (2)^{2} = 4
\]
\[
\|x\|_p + \|y\|_p = 1^2 + 1^2 = 2
\]

\[
\Rightarrow \|x + y\|_p = 4 > 2 = \|x\|_p + \|y\|_p
\]

So, the triangle inequality fails \(\Rightarrow\) not a norm.

\subsection*{(c) Does the triangle inequality hold for \( p = 0 \)?}

No. The expression:
\[
\|x\|_0 := \#\{i : x_i \neq 0\}
\]
counts the number of non-zero elements in \( x \). It is not a norm because:
\begin{itemize}
    \item It is not homogeneous: \( \|\alpha x\|_0 = \|x\|_0 \) if \( \alpha \neq 0 \)
    \item It violates the triangle inequality:
    \[
    \|x + y\|_0 \leq \|x\|_0 + \|y\|_0 \quad \text{may fail, e.g. overlapping support}
    \]
\end{itemize}
So \( \|\cdot\|_0 \) is not a norm.

\subsection*{(d) Show that for \( p \geq 1 \), the unit ball \( B_p := \{x \in \mathbb{R}^n : \|x\|_p \leq 1\} \) is convex}

Let \( x_1, x_2 \in B_p \), and \( 0 < \alpha < 1 \). Then:
\[
\|\alpha x_1 + (1 - \alpha)x_2\|_p \leq \alpha\|x_1\|_p + (1 - \alpha)\|x_2\|_p \leq \alpha + (1 - \alpha) = 1
\]

Hence, \( \alpha x_1 + (1 - \alpha)x_2 \in B_p \), so \( B_p \) is convex.

\subsection*{(e) Show that for \( 0 < p < 1 \), the unit ball \( B_p \) is not convex}

Let \( x_1 = (1, 0), x_2 = (0, 1) \Rightarrow \|x_1\|_p = \|x_2\|_p = 1 \Rightarrow x_1, x_2 \in B_p \)

Now, take the midpoint:
\[
z = \frac{1}{2}x_1 + \frac{1}{2}x_2 = \left(\frac{1}{2}, \frac{1}{2}\right)
\]
\[
\|z\|_p = \left(2 \cdot \left(\frac{1}{2}\right)^p\right)^{1/p} = (2^{1 - p})^{1/p} = 2^{\frac{1 - p}{p}} > 1 \quad \text{for } p < 1
\]
So \( z \notin B_p \), and \( B_p \) is not convex.

\subsection*{(f) Show that \( \|A\|_F^2 = \text{Tr}(A^\top A) = \text{Tr}(AA^\top) \)}

Let \( A \in \mathbb{R}^{m \times n} \), and define the Frobenius norm:
\[
\|A\|_F = \left( \sum_{i=1}^m \sum_{j=1}^n |a_{ij}|^2 \right)^{1/2}
\Rightarrow \|A\|_F^2 = \sum_{i=1}^m \sum_{j=1}^n |a_{ij}|^2
\]

Now observe:
\[
\text{Tr}(A^\top A) = \sum_{i=1}^n (A^\top A)_{ii} = \sum_{i=1}^n \sum_{k=1}^m A_{ki}^2 = \sum_{i=1}^m \sum_{j=1}^n a_{ij}^2 = \|A\|_F^2
\]
Similarly,
\[
\text{Tr}(AA^\top) = \|A\|_F^2
\]

\subsection*{(g) Provide a closed-form expression for the operator norm of \( A \in \mathbb{R}^{m \times n} \) w.r.t. \( p = 1 \)}

The operator norm induced by the \( \ell_1 \)-norm is:
\[
\|A\|_{1 \to 1} = \max_{x \neq 0} \frac{\|Ax\|_1}{\|x\|_1} = \max_{j=1,\dots,n} \sum_{i=1}^m |a_{ij}|
\]
i.e., the maximum absolute column sum.

\subsection*{(h) Provide a closed-form expression for the operator norm of \( A \in \mathbb{R}^{m \times n} \) w.r.t. \( p = \infty \)}

The operator norm induced by the \( \ell_\infty \)-norm is:
\[
\|A\|_{\infty \to \infty} = \max_{x \neq 0} \frac{\|Ax\|_\infty}{\|x\|_\infty} = \max_{i=1,\dots,m} \sum_{j=1}^n |a_{ij}|
\]
i.e., the maximum absolute row sum.

\subsection*{(i*) Provide a closed-form expression for the operator norm of \( A \) w.r.t. \( p = 2 \)}

The \( \ell_2 \to \ell_2 \) operator norm is the largest singular value of \( A \):
\[
\|A\|_{2 \to 2} = \sigma_{\max}(A) = \sqrt{\lambda_{\max}(A^\top A)}
\]

\subsection*{(j*) Show that the operator norm can equivalently be defined as:}
\[
\|A\|_{op} = \sup_{v \neq 0, v \in V} \frac{\|Av\|}{\|v\|} = \sup_{\|v\|=1, v \in V} \|Av\|
\]

\textbf{Proof:} For any non-zero vector \( v \), define \( u = \frac{v}{\|v\|} \Rightarrow \|u\| = 1 \), so:
\[
\frac{\|Av\|}{\|v\|} = \left\|A \left(\frac{v}{\|v\|} \right) \right\| = \|Au\|
\Rightarrow \sup_{v \neq 0} \frac{\|Av\|}{\|v\|} = \sup_{\|v\|=1} \|Av\|
\]

\subsection*{(k) Show submultiplicativity for square matrices \( A, B \in \mathbb{R}^{n \times n} \):}
\[
\|AB\| \leq \|A\| \cdot \|B\|
\]

Let \( x \in \mathbb{R}^n \) with \( \|x\| = 1 \). Then:
\[
\|ABx\| \leq \|A\| \cdot \|Bx\| \leq \|A\| \cdot \|B\| \cdot \|x\| = \|A\| \cdot \|B\|
\Rightarrow \sup_{\|x\| = 1} \|ABx\| \leq \|A\| \cdot \|B\|
\Rightarrow \|AB\| \leq \|A\| \cdot \|B\|
\]

\section*{2. Least Squares and Matrix Derivatives}

\subsection*{(a) Show that \( \nabla_\beta(z^\top \beta) = z \), where \( z, \beta \in \mathbb{R}^{n \times 1} \)}

We apply the identity for gradients of linear functions:
\[
\nabla_\beta(z^\top \beta) = \nabla_\beta(\beta^\top z) = z
\]
This follows from the fact that \( z^\top \beta \) is a scalar and the gradient of a scalar linear form is the coefficient vector.

\subsection*{(b) Show that \( \nabla_\beta(\beta^\top H \beta) = 2H\beta \), where \( H \in \mathbb{R}^{n \times n} \) is symmetric}

We use the identity for the gradient of a quadratic form:
\[
\nabla_\beta(\beta^\top H \beta) = (H + H^\top)\beta
\]
Since \( H \) is symmetric (\( H = H^\top \)), we get:
\[
\nabla_\beta(\beta^\top H \beta) = 2H\beta
\]

\subsection*{(c) Given a sample \( \{x_i\}_{i=1}^n \subset \mathbb{R} \), find}
\[
c_1^* = \arg \min_{c \in \mathbb{R}} \sum_{i=1}^n |x_i - c|
\]

The solution to minimizing the sum of absolute deviations is the \textbf{median}:
\[
c_1^* = \text{median}(x_1, x_2, \dots, x_n)
\]

\subsection*{(d) Given a sample \( \{x_i\}_{i=1}^n \subset \mathbb{R} \), find}
\[
c_2^* = \arg \min_{c \in \mathbb{R}} \sum_{i=1}^n (x_i - c)^2
\]

This is the classic least squares minimizer. Taking derivative and setting to zero:
\[
\frac{d}{dc} \sum_{i=1}^n (x_i - c)^2 = -2 \sum_{i=1}^n (x_i - c) = 0 \Rightarrow c_2^* = \frac{1}{n} \sum_{i=1}^n x_i
\]
So the minimizer is the \textbf{mean} of the data:
\[
c_2^* = \bar{x}
\]

\subsection*{(e*) Given a sample \( \{x_i\}_{i=1}^n \subset \mathbb{R}^d \), find}
\[
c_1^* = \arg \min_{c \in \mathbb{R}^d} \sum_{i=1}^n \|x_i - c\|
\]

There is no closed-form solution for this problem (sum of Euclidean distances). The solution is known as the \textbf{geometric median}, which must be computed using iterative algorithms (I read about Weiszfeld's algorithm).

\subsection*{(f) Given a sample \( \{x_i\}_{i=1}^n \subset \mathbb{R}^d \), find}
\[
c_2^* = \arg \min_{c \in \mathbb{R}^d} \sum_{i=1}^n \|x_i - c\|^2
\]

This is the multivariate least squares problem. The objective is minimized when \( c \) is the \textbf{mean} of the vectors:
\[
c_2^* = \frac{1}{n} \sum_{i=1}^n x_i
\]
\section*{3. Regularized Least Squares Regression}

We define Ridge and Lasso regression:
\[
\hat{\beta}_{\text{Ridge}} = \arg\min_\beta \|X\beta - y\|_2^2 + \lambda_{\text{Ridge}} \|\beta\|_2^2
\]
\[
\hat{\beta}_{\text{Lasso}} = \arg\min_\beta \|X\beta - y\|_2^2 + \lambda_{\text{Lasso}} \|\beta\|_1
\]

\subsection*{(a) Show that the closed-form solution to equation (2) is:}
\[
\hat{\beta}_{\text{Ridge}} = (X^\top X + \lambda I)^{-1} X^\top y
\]

\textbf{Solution:}

We start with the Ridge objective function:
\[
J(\beta) = \|X\beta - y\|_2^2 + \lambda \|\beta\|_2^2 = (X\beta - y)^\top (X\beta - y) + \lambda \beta^\top \beta
\]

Take the gradient with respect to \( \beta \):
\[
\nabla_\beta J = 2X^\top (X\beta - y) + 2\lambda \beta
\]

Set gradient to zero:
\[
2X^\top X\beta - 2X^\top y + 2\lambda \beta = 0
\Rightarrow X^\top X\beta + \lambda \beta = X^\top y
\Rightarrow (X^\top X + \lambda I)\beta = X^\top y
\]

Solving for \( \beta \), we get:
\[
\hat{\beta}_{\text{Ridge}} = (X^\top X + \lambda I)^{-1} X^\top y
\]

\subsection*{(b) Provide sufficient conditions for the invertibility of \( X^\top X + \lambda I \)}

\textbf{Answer:}

- \( X^\top X \) is a symmetric positive semi-definite matrix (PSD).
- Adding \( \lambda I \), where \( \lambda > 0 \), results in a strictly positive definite matrix:
\[
X^\top X + \lambda I \succ 0
\]

\textbf{Sufficient condition:}
\[
\lambda > 0
\]
This ensures that \( X^\top X + \lambda I \) is strictly positive definite and hence invertible.

If \( X^\top X \) is not full rank (e.g. if \( X \) is not full column rank), \( X^\top X \) may be singular — but the addition of \( \lambda I \) ensures it becomes full rank and invertible as long as \( \lambda > 0 \).

\end{document}
